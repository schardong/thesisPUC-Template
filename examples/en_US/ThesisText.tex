%%% -*- coding: utf-8 -*-
\newpage

\chapter{Introduction}
\label{chap:introduction}

This is a simple, introductory text demonstrating how to use the PUC-Rio's thesis/dissertations \LaTeX~class. Here are some citations~\cite{Lee2010},~\cite{Hummel2013},~\cite{Dupacova2003}. In text citations, according to~\cite{Armstrong2013} work the same way.

In summary, the contributions of this class are:

\begin{itemize}
  \item A more up-to-date \LaTeX~template for thesis/dissertations for the academic community of PUC-Rio's;
  \item A way to centralize changes and increase the reach of fixes and adjustments.
\end{itemize}

Table~\ref{tab:cmg-results} and Figure~\ref{fig:explicit-endoding-szafir} present ways to include tables and figures in the text. Figure~\ref{fig:explicit-encoding-tllamp-inc} shows how to include subfigures. Section~\ref{sec:organization} is just a dummy section.

\section{Thesis Structure}
\label{sec:organization}

\subsection{Example subsection}

\begin{table}[H]
  \centering
  \caption[Errors for scenarios selected using the industry standard approach.]{Scenarios and errors ($\times 10^{10}$) for the industry standard approach.}
  \begin{tabular}{llrrr}
    \hline
    \textbf{Property}       & \textbf{Percentile}       & \textbf{Scenario} & \textbf{SSE}      & \textbf{MSE}   \\ \hline
    \multirow{12}{*}{$N_p$} & \multirow{4}{*}{P$_{10}$} & 172               & 1190.0          & 30.5           \\
                            &                           & 88                & 663.0          & 17.0           \\
                            &                           & 122               & 1660.0          & 42.6           \\
                            &                           & \textbf{36}       & \textbf{312.0} & \textbf{7.9}   \\ \cline{2-5} 
                            & \multirow{4}{*}{P$_{50}$} & 131               & 162.0          & 4.1            \\
                            &                           & \textbf{4}        & \textbf{162.0} & \textbf{4.1}   \\
                            &                           & 26                & 786.0          & 20.1           \\
                            &                           & 90                & 236.0          & 6.0            \\ \cline{2-5} 
                            & \multirow{4}{*}{P$_{90}$} & 96                & 975.0          & 25.0           \\
                            &                           & 100               & 690.0          & 17.7           \\
                            &                           & 127               & 925.0          & 23.7           \\
                            &                           & \textbf{132}      & \textbf{526.0} & \textbf{13.5}  \\ \hline
    \multirow{12}{*}{$W_p$} & \multirow{4}{*}{P$_{10}$} & \textbf{23}       & \textbf{7640.0} & \textbf{196.0} \\
                            &                           & 115               & 42000.0          & 1080.0         \\
                            &                           & 10                & 26900.0          & 690.0          \\
                            &                           & 19                & 27600.0          & 707.0          \\ \cline{2-5} 
                            & \multirow{4}{*}{P$_{50}$} & 29                & 14200.0          & 364.0          \\
                            &                           & 153               & 6260.0          & 161.0          \\
                            &                           & \textbf{84}       & \textbf{5840.0} & \textbf{150.0} \\
                            &                           & 88                & 11200.0          & 288.0          \\ \cline{2-5} 
                            & \multirow{4}{*}{P$_{90}$} & 37                & 14300.0          & 367.0          \\
                            &                           & 57                & 22300.0          & 571.0          \\
                            &                           & 187               & 14000.0          & 358.0          \\
                            &                           & \textbf{28}       & \textbf{2210.0} & \textbf{56.7}  \\ \hline
  \end{tabular}
  \label{tab:cmg-results}
\end{table}

\begin{figure}[H]
  \centering
  \includegraphics[width=\textwidth]{explicit_encoding}
  \caption{Juxtaposition, Superposition and Explicit Encoding. Image taken from the work of~\cite{Szafir2018}.}
  \label{fig:explicit-endoding-szafir}
\end{figure}

\begin{figure}[H]
  \centering
  \begin{subfigure}[t]{0.75\textwidth}
    \centering
    \includegraphics[width=\columnwidth]{WP-tllamp}
    \caption{Superposition of times in the Time-lapsed LAMP chart.}
    \label{fig:superposition-tllamp-inc}
  \end{subfigure}
  ~
  \begin{subfigure}[t]{0.75\textwidth}
    \centering
    \includegraphics[width=\columnwidth]{WP-tllamp-linear-inc-glyph}
    \caption{Explicit encoding of time in the Time-lapsed LAMP chart. Starting times encoded with smaller glyphs.}
    \label{fig:ee-tllamp-inc}
  \end{subfigure}
  \caption{Comparison between different time encodings on the Time-lapsed LAMP chart of $W_p$. Superposition~(\ref{fig:superposition-tllamp-inc}) and Explicit Encoding~(\ref{fig:ee-tllamp-inc}) from small to large glyphs.}
  \label{fig:explicit-encoding-tllamp-inc}
\end{figure}

\subsection{Sample theorems, lemmas and proofs}

\begin{theorem}
Let $f$ be a function whose derivative exists in every point, then $f$ is 
a continuous function.
\end{theorem}
 
\begin{theorem}[Pythagorean theorem]
\label{pythagorean}
This is a theorema about right triangles and can be summarised in the next 
equation 
\[ x^2 + y^2 = z^2 \]
\end{theorem}
 
And a consequence of theorem \ref{pythagorean} is the statement in the next 
corollary.
 
\begin{corollary}
There's no right rectangle whose sides measure 3cm, 4cm, and 6cm.
\end{corollary}

\begin{lemma}[Unimodularity preserving matrix operations]
The following matrix operations, called unimodular operations, preserve the unimodularity property from an already unimodular matrix:
\begin{enumerate}
    \item Interchange two columns
    \item Add an integer multiple of a column to another column
    \item Multiply a column by $-1$
    \item Transpose
\end{enumerate}
\end{lemma}

\begin{lemma}
\label{line-segs}
Given two line segments whose lengths are $a$ and $b$ respectively there 
is a real number $r$ such that $b=ra$.
\end{lemma}
 
\begin{proof}
To prove Lemma~\ref{line-segs} by contradiction try and assume that the statemenet is false,
proceed from there and at some point you will arrive to a contradiction.
\end{proof}

%%% Local Variables:
%%% mode: latex
%%% TeX-master: thesis.tex
%%% End:
